%%%%%%%%%%%%%%%%%%%%%%%%%%%%%%%%%%%%%%%%%
% Medium Length Graduate Curriculum Vitae
% LaTeX Template
% Version 1.1 (9/12/12)
%
% This template has been downloaded from:
% http://www.LaTeXTemplates.com
%
% Original author:
% Rensselaer Polytechnic Institute (http://www.rpi.edu/dept/arc/training/latex/resumes/)
%
% Important note:
% This template requires the res.cls file to be in the same directory as the
% .tex file. The res.cls file provides the resume style used for structuring the
% document.
%
%%%%%%%%%%%%%%%%%%%%%%%%%%%%%%%%%%%%%%%%%

%----------------------------------------------------------------------------------------
%	PACKAGES AND OTHER DOCUMENT CONFIGURATIONS
%----------------------------------------------------------------------------------------

\documentclass[margin]{res} % Use the res.cls style, the font size can be changed to 11pt or 12pt here

\usepackage{helvet} % Default font is the helvetica postscript font
%\usepackage{newcent} % To change the default font to the new century schoolbook postscript font uncomment this line and comment the one above

\setlength{\textwidth}{5.1in} % Text width of the document

\begin{document}

%----------------------------------------------------------------------------------------
%	NAME AND ADDRESS SECTION
%----------------------------------------------------------------------------------------

\moveleft.5\hoffset\centerline{\huge\bf Christopher Richardson} % Your name at the top

\moveleft\hoffset\vbox{\hrule width\resumewidth height 1pt}\smallskip % Horizontal line after name; adjust line thickness by changing the '1pt'

\moveleft.5\hoffset\centerline{1163 West Peachtree St NE, Apt 1515} % Your address
\moveleft.5\hoffset\centerline{Atlanta, GA 30309}
\moveleft.5\hoffset\centerline{404-819-8257}
\moveleft.5\hoffset\centerline{crichardson332@gmail.com}

%----------------------------------------------------------------------------------------

\begin{resume}

%%----------------------------------------------------------------------------------------
%%	OBJECTIVE SECTION
%%----------------------------------------------------------------------------------------
%
%\section{OBJECTIVE}
%
%To secure a full-time position designing autonomy algorithms in the field of robotics.

%----------------------------------------------------------------------------------------
%	INTERESTS SECTION
%----------------------------------------------------------------------------------------

\section{INTERESTS}

Autonomy for robotic systems; assured/robust autonomy; multi-vehicle control and coordination; flight controls; stochastic/Bayesian estimation; path-planning; trajectory optimization; machine learning; deep learning; neural nets

%----------------------------------------------------------------------------------------
%	COMPUTER SKILLS SECTION
%----------------------------------------------------------------------------------------

\section{SOFTWARE \\ SKILLS}

{\sl Expert:} \\
C++, python, git, cmake, subversion, Matlab, Simulink, object-oriented software design, unit testing \\ \\
{\sl Experienced:} \\
C, ROS, bash/shell scripting, LLDB (LLVM debugger) make/makefiles,  GDB, gtest, Boost, swig, doxygen, MavLink \\ \\
{\sl Libraries:} \\
libstdc++, libc++, Eigen, GeographicLib, Boost, VTK, Protobuf, ZeroMQ

%----------------------------------------------------------------------------------------
%	PROFESSIONAL EXPERIENCE SECTION
%----------------------------------------------------------------------------------------

\section{EXPERIENCE}

{\sl Research Engineer I} \hfill July 2017-Present \\
Georgia Tech Research Institute, Atlanta, GA
\begin{itemize} \itemsep -2pt % Reduce space between items
\item Collaborative Autonomy Group
\item Multi-vehicle and swarm autonomy algorithm development
\vspace{-2mm}
	\begin{itemize}
	\item Formation and coordination using gemoetrically defined formations in a leader-follower model
	\item Path planning using stochastic algorithms (RRT) and geometric optimization (Dubins)
	\item Simple PID controls for following defined paths (racetracks, staging orbits, etc.)
	\end{itemize}
\item Software architecture design, development, and tesing
\vspace{-2mm}
	\begin{itemize}
	\item Developer on GTRI's open-source multi-vehicle simulation tool SCRIMMAGE
	\item Ported linux-based SCRIMMAGE to macOS
		\begin{itemize}
		\item Used homebrew package manager for dependencies
		\item Created and maintained several homebrew packages for dependencies not available before
		\item Maintained fork of JSBSim, VTK, and rapidxml for SCRIMMAGE macOS port
		\end{itemize}
	\item Helped modernize and optimize SCRIMMAGE's CMake-based build system
	\item Integrated SCRIMMAGE API with ROS autonomy stack

	\end{itemize}
\item Integration and flight testing of UAV swarms
\vspace{-2mm}
	\begin{itemize}
	\item Cross-compiled autonomy onto ODROID hardware for HiL and flight testing
	\item Vehicles operated through GTRI's Swarm Operator Interface (SOI)
	\item Flight testing of four autonomous helicopters controlled by Georgia Tech GUST autopilot (Eric Johnson)
	\end{itemize}
\end{itemize}

{\sl Guidance, Nagivation, and Control Engineer} \hfill June 2016-June 2017 \\
Aurora Flight Sciences, Cambridge, MA
\begin{itemize} \itemsep -2pt % Reduce space between items
\item Development of multi-vehicle task allocation and path planning tool for small business technology transfer (STTR) program
\vspace{-2mm}
	\begin{itemize}
	\item Partnered with MIT to demonstrate algorithms in a scenario involving a satellite constellation
	\item Developed simulation in Python and ported algorithms from C++
	\item Task allocation using consensus-based bundle algorithm (CBBA)
	\end{itemize}
\item Development and implementation of autopilot for Darpa's ALIAS program
\vspace{-2mm}
	\begin{itemize}
	\item Design of minimally-invasive robotic system with perception and actuation that adds autonomy to existing aircraft
	\item Designed state estimation system fusing data from inertial navigation with air data and magnetometer measurements from perception system
	\item Involved in full design process, including simulation and analysis, hardware-in-the-loop simulation, and flight test
	\end{itemize}
\item Control system design and analysis, including both time-domain and frequency-domain methods
\item Gain scheduling and optimization using control design tools in Matlab and Simulink
\end{itemize}

{\sl Graduate Research Assistant} \hfill January-May 2016\\
Georgia Institute of Technology, Atlanta, GA
\begin{itemize} \itemsep -2pt % Reduce space between items
\item Performed online parameter identification for helicopter vibration control
\item Researched various algorithms including Recursive Least Squares and Kalman filter
\item Tested algorithms in Matlab simulation environment and verified results with nonlinear simulation in FlightGear
\end{itemize}

{\sl Robotics Modeling and Simulation Co-op} \hfill January-July 2015 \\
Jet Propulsion Laboratory, Pasadena, CA

\begin{itemize} \itemsep -2pt % Reduce space between items
\item Developed multibody dynamics simulation of a Mars helicopter
\item Created C++ models for aerodynamics and helicopter mechanisms
\item Developed fully configurable, easy-to-use Python simulation for ease in testing various helicopter designs and rotor mechanisms
\vspace{-2mm}
	\begin{itemize}
	\item Simulation used by Mars helicopter design team to assist in gaining a fundamental understanding of helicopter 	dynamics in a Martian environment
	\end{itemize}
\end{itemize}

{\sl Graduate Student Intern} \hfill Summer 2014 \\
Jet Propulsion Laboratory, Pasadena, CA
\begin{itemize} \itemsep -2pt % Reduce space between items
\item Assisted in the design of a low cost heliostat for concentrated solar power plants
\item Contributed to trade studies and demonstrated lower cost for alternatives with standalone power analysis
\item Developed model for sun tracking algorithm and control system for heliostat
\end{itemize}

{\sl Undergraduate Design Engineer} \hfill Spring/Summer 2013 \\
Georgia Institute of Technology, Atlanta, GA
\begin{itemize} \itemsep -2pt % Reduce space between items
\item Founding member of the RECONSO team, winner of the 8\textsuperscript{th} iteration of the University Nanosat Program
\item Performed requirements analysis and trade studies for various subsystems
\end{itemize}

%----------------------------------------------------------------------------------------
%	EDUCATION SECTION
%----------------------------------------------------------------------------------------

\section{EDUCATION}
{\sl Master of Science,} Aerospace Engineering \hfill May 2016 \\
Georgia Institute of Technology \\
GPA: 3.91/4.00

{\sl Bachelor of Science,} Highest Honors, Aerospace Engineering \hfill May 2013 \\
Georgia Institute of Technology \\
GPA: 3.91/4.00


%----------------------------------------------------------------------------------------
%	ENGINEERING SKILLS SECTION
%----------------------------------------------------------------------------------------

%\section{ENGINEERING \\ SKILLS}

%Knowledge of space systems design and spacecraft dynamics, control systems design, propulsion systems, flight %dynamics, structural mechanics/dynamics, aerodynamics, thermodynamics

%----------------------------------------------------------------------------------------

\end{resume}
\end{document}
